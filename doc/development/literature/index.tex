\documentclass{article}

\usepackage{hyperref}

\usepackage{natbib}
\bibliographystyle{plainnat}
\usepackage{bibentry}
\nobibliography*

%%%%%%%%%%%%%%%%%%%%%%%%%%%%%%%%%%%%%%%%%%%%%%%%%% {{{
\usepackage{usebib}
%% prints out the info for a citation key:
%%     \printarticle{Author00}
\newcommand{\printarticle}[1]{\citeauthor{#1}, ``\usebibentry{#1}{title}'' (\usebibentry{#1}{year})}
\newcommand{\paracite}[1]{\paragraph{\usebibentry{#1}{title}~\citet{#1}}}
% \bibentry{#1}
%%%%%%%%%%%%%%%%%%%%%%%%%%%%%%%%%%%%%%%%%%%%%%%%%% }}}

\bibinput{biblio}

\def\printindex{}

\begin{document}

\section{Bret Victor}

\begin{itemize}
\item \url{http://worrydream.com/MediaForThinkingTheUnthinkable/note.html}
\item \url{http://worrydream.com/MediaForThinkingTheUnthinkable/}
\item \url{http://worrydream.com/MagicInk/}
\item \url{http://worrydream.com/TheHumaneRepresentationOfThought/note.html}
\end{itemize}

\section{Alan Kay}

\url{https://www.youtube.com/watch?v=YyIQKBzIuBY}, \href{http://www.tele-task.de/archive/video/flash/14029/}{HPI}

\section{Design of electronic books}

\paracite{marshall2009-reading}

\paracite{pearson2013-digital-reading}

\section{Software engineering}

\paracite{rosenberg2008-dreaming-code}

\section{Workflow}

In \href{http://ahiddendiscourse.com/blog/2013/02/17/the-cognitive-basis-for-academic-workflows/}{The cognitive basis for academic workflows},
Lisa D. Harper looks at sensemaking models as a way to understand academic
workflows.


\bibliography{biblio}

\end{document}
